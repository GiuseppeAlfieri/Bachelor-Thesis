\begingroup 
\titleformat{\chapter}
{\normalfont\Huge\bfseries\centering} %shape
%{\centering} % format
{}% label
{} % sep
{}  

\chapter*{Abstract}


I generatori di immagini da prompt testuali, quali \emph{Midjourney}~\cite{Midjourney}, sono in grado di creare, partendo da descrizioni testuali, 
immagini di alta qualità che siano quanto più fedeli possibile alla \emph{semantica} del testo. 
Queste tecnologie si avvalgono di modelli generativi che, a loro volta, sottendono l'impiego di reti neurali.

In questa tesi si fornirà, preliminarmente, una panoramica generale 
della modellazione generativa, declinata al contesto della generazione di immagini, 
congiuntamente ad una visione d'insieme delle famiglie di modelli generativi più in auge in letteratura.

Successivamente verrà discussa la \emph{ratio} sottesa a una delle tecnologie, nella pletora di modelli generativi, che 
rendono possibile la “magia” di generare immagini: i modelli probabilistici di diffusione del rumore (\emph{Denoising Diffusion Probabilistic Models}) (DDPM).


\endgroup
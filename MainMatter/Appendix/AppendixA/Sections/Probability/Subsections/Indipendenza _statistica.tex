\subsection{Indipendenza statistica}


\begin{Mybox}
    \begin{definizione}[Indipendenza statistica]
       Due vettori aleatori $\mathbf{X},\mathbf{Y}$ sono \emph{statisticamente indipendenti} se e solo se
    \begin{equation}
        p(\mathbf{x},\mathbf{y})=p(\mathbf{x})p(\mathbf{y})\label{eq:indipendenza}
    \end{equation}
    \end{definizione}
\end{Mybox}


\noindent Intuitivamente, due vettori aleatori $\mathbf{X}$ e $\mathbf{Y}$ sono indipendenti se la 
conoscenza dei valori assunti da uno dei due non dà nessuna informazione sui valori assunti dall'altro. 



Qualora i vettori $\mathbf{X}$ e $\mathbf{Y}$ siano statisticamente indipendenti, risulta che:
\begin{align}
    p(\mathbf{y}| \mathbf{x}) & = p(\mathbf{y}) \\
    p(\mathbf{x}| \mathbf{y}) & = p(\mathbf{x}) \\
    \mathbb{V}ar[\mathbf{X}+\mathbf{Y}] & = \mathbb{V}ar[\mathbf{X}]+\mathbb{V}ar[\mathbf{Y}] \\
    \text{Cov}[\mathbf{X},\mathbf{Y}]   & = \bm{0}
\end{align}
